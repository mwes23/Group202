\documentclass[12pt,letterpaper]{article}
\begin{document}

\title{COLLEGE OF COMPUTING AND INFORMATION SCIENCES\\ DEPARTMENT OF COMPUTER SCIENCE\\ SCHOOL OF COMPUTING AND INFORMATICS TECHNOLOGY\\}
\maketitle
\title{NAME:	     OKIRING PAUL\\
STUDENTS NUMBER:	 214016765\\
REGISTRATION NUMBER: 14/U/13973/EVE}
\maketitle

\section{INTRODUCTION}

     Model checking has emerged as a powerful method for the formal verification of programs. Temporal logics such as CTL (computational tree logic) and CTL* are widely used to specify programs because they are expressive and easy to understand. Given an abstract model of a program, a model checker (which typically implements the acceptance problem for a class of automata) verifies whether the model meets a given specification. A conceptually attractive method for solving the model checking problem is by reducing it to the solution of (a suitable subclass of) parity games. These are a type of two player infinite game played on a finite graph.
\section{PROBLEM STATEMENT}
Given a model of a system, exhaustively and automatically check whether this model meets a given specification. Typically, one has hardware or software systems in mind, whereas the specification contains safety requirements such as the absence of deadlocks and similar critical states that can cause the system to crash. Model checking is a technique for automatically verifying correctness properties of finite-state systems.
In order to solve such a problem algorithmically, both the model of the system and the specification are formulated in some precise mathematical language. To this end, the problem is formulated as a task in logic, namely to check whether a given structure satisfies a given logical formula. This general concept applies to many kinds of logics and suitable structures. A simple model checking problem is verifying whether a given formula in the propositional logic is satisfied by a given structure.

\section{BACKGROUND}

\section{MAIN OBJECTIVES}
 To show the connexions between the temporal logics CTL and / or CTL*, automata, and games.
\subsection{OBJECTIVES}
  \begin{enumerate}
    \item Representing CTL / CTL* as classes of alternating tree automata\\
    \item Inter-translation between CTL / CTL* and classes of alternating tree automata\\
    \item Using B¨uchi games and other subclasses of parity games to analyze the CTL / CTL* model checking problem\\
    \item Efficient implementation of model checking algorithms\\
    \item Application of the model checker to higher-order model checking\\
  \end{enumerate}
\section{Scope}
\section{Descriptions}
\section{Significances}
\section{Methodologies}
\section{References}
\end{document} 